\documentclass[10pt]{article}
\usepackage{../be-my-concrete}
\usepackage{../be-my-geometry}
\usepackage{../extra-topology}

\begin{document}
\makedz{9}
\tableofcontents
\begin{tasks}
	\addcontentsline{toc}{section}{Задача 1} \item  Существует ли отношение эквивалентности на отрезке, факторпространство по которому гомеоморфно объединению 3 отрезков с общим концом? Строго обоснуйте ответ.
	
	\begin{proof}[Решение]
		На отрезке $AE$ выберем точки $B, C, D$ так, чтобы $AB \subset AC \subset AD \subset AE.$ Введем такое отображение $$f \colon AE \to AE, \qquad \left.\begin{aligned}
			AB \mapsto AB\\
			BC \mapsto BC\\
			CD \mapsto BC\\
			DE \mapsto DE
		\end{aligned} \right.$$
		
		\begin{figure}[h]
			\centering
			\begin{asy}
				size(9cm);
				defaultpen(fontsize(10pt));
				draw((0,0)--(4,0), linewidth(bp));
				dot("$A$", (0,0), N);
				dot("$B$", (1,0), N);
				dot("$C$", (2,0), N);
				dot("$D$", (3,0), N);
				dot("$E$", (4,0), N);
				
				draw((5,0)--(6,0),arrow=ArcArrow(SimpleHead));
				label("$f$", (5.5, 0.05), align=N);
				
				draw((7,0)--(8.5,0)--(8.5,1.5)--(8.5,0)--(10,0), linewidth(bp));
				
				dot("$f(A)$", (7,0), N);
				dot("$f(B)=f(D)$", (8.5,0), S);
				dot("$f(C)$", (8.5,1.5), NE);
				dot("$f(E)$", (10,0), N);
				
			\end{asy}
			\caption{Отрезок $AE$ и $f(AE)$.}
		\end{figure}
		
		Такое отображение является непрерывной сюръекцией \footnote{Прообразами открытых являются либо одно, либо два открытых множества в $AE$; а объединение двух открытых множеств открыто.}. Отрезок $AE$ является компактом, а $f(AE)$ хаусдорфово. Тогда \[\frac{AE}{\sim_f} \cong \text{объединение 3 отрезков с общим концом}.\]  
		
		\ans{Да.}
	\end{proof}
	\pagebreak
	\coffeestainA{0.15}{0.5}{-25}{2.3cm}{1.3cm}
	\vspace{-6mm}
	
	\addcontentsline{toc}{section}{Задача 2} \item Факторпространство топологического пространства по некоторому отношению эквивалентности является хаусдорфовым тогда и только тогда, когда любые два класса эквивалентности обладают непересекающимися насыщенными открытыми окрестностями. Докажите.
	
	Останется ли данное утверждение верным, если удалить слово "открытыми"?
	
	Напомним, что окрестность подмножества $A\subset X$ топологического пространства $X$ — это такое подмножество $O\subset A$, в котором есть открытое подмножество $U\supset A$, так что $A\subset U\subset O \subset X$.
	
	\begin{proof}[Решение]
		\fbox{$\implies$} \hspace{1ex} Пусть $\frac X \sim$ хаусдорфово. Тогда докажем, что для любых двух классов эквивалентности, существуют две непересекающиеся, насыщенные открытые окрестности.
		
		По определению хаусдорфовости $\rfrac X \sim$ для любых $[x_1], [x_2]$ существуют открытые $U, V$ такие, что \[U \ni [x_1], V \ni [x_2], \quad U \cap V = \emptyset, \quad U, V \in \T_{\rfrac X \sim}.\]
		
		И пусть $\pi\colon X \to \rfrac X \sim$ есть каноническая проекция на факторпространство. Тогда докажем, что $\pi^{-1}(U)$ и $\pi^{-1}(V)$ -- подходящие окрестности. 
		\begin{conditions}
			\item \textit{Открытость.} Следует из непрерывности $\pi$  \[U, V \in \T_{\rfrac X \sim} \implies \pi^{-1}(U), \pi^{-1}(V) \in \T_X.\]
			\item \textit{Пустое пересечение.} Пусть $\pi^{-1}(U)\cap \pi^{-1}(V) \ni x$. Тогда $\pi(x) \in U$ и $\pi(x) \in V$. Получаем противоречие с $U \cap V = \emptyset$.
			\item \textit{Насыщенность.} Пусть $a \in \pi^{-1}(U)$ и $a \sim b$. Тогда \[\underbrace{\pi(a)}_{=\pi(b)} \in U \implies \pi^{-1}(U) \ni b.\] Аналогичное рассуждение проделывается для $V$.
		\end{conditions} 
		
		\fbox{$\impliedby$} Пусть $x_1$ и $x_2$ обладают непересекающимися открытыми насыщенными окрестностями $U, V$ соответственно. Докажем, что $[x_1]$, $[x_2]$ обладают непересекающимися открытыми окрестностями. 
		Докажем, что $\pi(U)$, $\pi(V)$ подходят.
		\begin{conditions}
			\item \textit{Открытость.} Следует из определения фактортопологии. 
			\item \textit{Пустое пересечение.} Пусть $x \in \pi(U)\cap \pi(V)$ Тогда $\pi^{-1}(x) \in U \cap V$, что противоречит $U \cap V = \emptyset.$ 
		\end{conditions}
		Таким образом $\rfrac X \sim$ хаусдорфово, так как $[x_1] \in pi(U)$, $[x_2] \in \pi(V).$ 
		
		\textit{Утверждение задачи не останется верным. Существует контрпример.}
	\end{proof}
	
	\addcontentsline{toc}{section}{Задача 3} \item Проективная прямая $\R P^1$ вкладывается в проективную плоскость $\R P^2$ следующим образом: точке из $\R P^1$ с однородными координатами $[x_0:x_1]$ соответствует точка $[x_0:x_1:0]\in \R P^2$. Докажите, что дополнение $\R P^2\setminus \R P^1$ гомеоморфно плоскости $\R^2$ со стандартной топологией.
	
	\begin{proof}
		[Решение] 
		$\R P^2 \setminus \R P^1 = \{[x_0:x_1:x_2]\such x_2 \neq 0\}.$ Тогда рассмотрим $$f \colon \R P^2 \setminus \R P^1 \hookrightarrow R^2, \quad [x_0:x_1:x_2] \mapsto \left( \frac{x_0}{x_2}, \frac{x_1}{x_2}\right)\footnotemark.$$
		\footnotetext{Отображение $f$ корректно определено, так как $x_2 \neq 0$.}
		Проверим что оно является сюръективным и инъеквтиным, а следовательно биективным.
		\begin{conditions}
			\item \textit{Инъективность.} Пусть $f([x_0:x_1:x_2]) = f([\tilde x_0 : \tilde x_1 : \tilde x_2])$. Тогда \[\frac{x_0}{x_2} = \frac{\tilde x_0}{\tilde x_2} = \frac{\lambda \tilde x_0}{x_2}, \quad \frac{x_1}{x_2} = \frac{\tilde x_1}{\tilde x_2}  = \frac{\lambda \tilde x_1}{x_2},\qquad \exists \lambda \in \R\setminus\{0\}: \quad \lambda \tilde x_2 = x_2.\]
			Ну и понятно, что \[[x_0: x_1:x_2] = [\lambda x_0:\lambda x_1 :\lambda x_2].\]
			\item \textit{Сюръективность.} \[f^{-1}((a, b)) \ni [a : b : 1].\]
		\end{conditions}
		Осталось проверить существование и непрерывность обратного.
		
		Определим отображение $$g \colon \R^2 \to \R P^2 \setminus \R P^1, \quad (a,b) \mapsto [a:b:1].$$
		\[(f\circ g)(a,b) = f([a:b:1]) = (a,b).\]
		\[(g\circ f)([a:b:1]) = g(a,b) = [a:b:1].\]
		
		Получаем, что $f$ и $g$ обратны друг другу, а также $f, g$ являются непрерывными. Таким образом $f$ -- гомеоморфизм пространств $\R P^2 \setminus \R P^1$ и $\R^2$. 

	\end{proof}
	
	%\addcontentsline{toc}{section}{Задача 4} \item Определите подпространство в $\C P^2$, гомеоморфное пространству $\R P^2$.
\end{tasks}


\end{document}